\chapter{Peer-to-peer Networks and Darknets}


\section{Peer-to-peer Networks}
In this section a short history of the internet leading to the development and spreading of peer-to-peer network architecture is given. It should provide a basic understanding of differences and advantages of P2P network design.

\subsection{From classical client-server architecture to distributed networks}
The internet emerged from several military and sience research networks, with ARPANET as the most commonly known of them. It was planed and designed as a decentralized telecomunication network resiliant to outages. Although its resiliance on a network level, most its services have still a centralized structure. On failure not the whole network will fail, but a service, as a website like \emph{www.tu-darmstadt.de} or its email system, can come unavailable quite easily.

This originates from the in most protocols and services used client-server architecture. The clients tries to find a single server to send its request to. The server processes the request and sends the response back. If in the time between the server is chosen by the client and it sends the response the server fails, the reqeust fails too. There are several methods in protocol and network design to prevent such failures. Avoidance of single-point-of-failures, redundancy, load balancing where requests are distributed to several servers are the common ones. All of them scale very badly and therefor are highly expensive.


\subsection{P2P Overlays}



\section{Darknets: privacy preserving P2P overlays}



\section{Darknet characteristics and resulting challenges}



\section{metrices for routing evaluation}



\section{Measurement methods for networks}


\subsection{Applicability on darknets}



\section{Survey of previous darknets}


